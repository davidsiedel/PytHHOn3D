% \documentclass{article}
\documentclass[fleqn]{article}

\usepackage{notations}
\usepackage{amsfonts}
\usepackage{amsmath}
\usepackage{relsize}
\usepackage{geometry}


\geometry{
 a4paper,
 total={170mm,257mm},
 left=20mm,
 top=20mm,
 }

% ======================================================================================================================
% NOTATIONS
% ======================================================================================================================

% ======================================================================================================================
% NOTATIONS
% ======================================================================================================================

% \setlength{\mathindent}{0pt}
\setlength{\parindent}{0em}
\setlength{\parskip}{1em}
\renewcommand{\baselinestretch}{1.0}

\title{\texttt{PytHHOn3D} documentation}
\author{David Siedel \and Olivier Fandeur \and  Thomas Helfer}

\date{01/08/2020}


% ======================================================================================================================
% DOCUMENT
% ======================================================================================================================

\begin{document}

  \maketitle

  \section{The HHO method}

    \subsection{Principle}

      The HHO method is a so-called "non-conformal method" (as opposed to conformal ones, among which is the Galerkin method). The main feature of non-conformal method lies in the foundation of the definition of the problem \eqref{eq_weak_problem}

      The HHO method, as well as the FE method, consists in discretizing \eqref{eq_weak_problem} both geometrically and functionally in order to actually compute approximated solutions that are not analytically reachable.
      \par
      Contrary to the FE method, the HHO method does not consider finding the solution in $\Hone$, but not in a richer space, namely the broken Sobolev space. Given, a mesh $\Mesh(\Omega)$ of $\Omega$ (see \ref{sec_mesh} for further description of what a mesh is within the framework of the HHO method), one defines the broken Sobolev space in the following way :
      % Since $\Omega$ is a continuum and $\Hone$ is of infinite dimension, discretization consists in both partitioning $\Omega$ into simplistic shapes called elements, and in approximating $\Hone$ with a polynomial space $\mathbb{P}_k^d(\Omega;\Rd) \subset \Hone$.
      % \par
      % % In the FE method, discretization in space is achieved by introducing a mesh of $\Omega$, and discretization of $\Hone$ is achieved by seeking $\tensori{u}$ the unknow of \eqref{eq_weak_problem} in $V \subset \Hone$ a subset of $\Hone$ of finite dimension. A convinient choice for $V$ is $\mathbb{P}_k^d(\Omega;\Rd)$ the set of $d$-variate polynomials of order $k$.
      % % \newline
      % The FE method uses Lagrange polynomials as a basis for $\mathbb{P}_k^d(\Omega;\Rd)$; such polynomials are intrinsically linked with the notion of mesh since they are the set of polynomials that take the value $1$ at a given node of the mesh, and that are $0$ elsewhere. The unknown at a given node of the mesh is then the value of its corresponding coordinate in the Lagrange polynomial basis, and elsewhere in $\Omega$, it is the interpolation of all the nodes value in a given element. The polynomial basis is then directly defined by the mesh of $\Omega$, and more complex shapes than triangles or quadrangles when $d = 2$, and tetrahedra, hexahedra or prisms when $d = 3$ are not practicable, since the Jacobian of the transformation of an element is not easily defined otherwise.
      % \par
      % Since $\mathbb{P}_k^d(\Omega;\Rd) \subset \Hone$ spans continuous (or conformal) polynomials, the displacement at a given point of the mesh is the interpolation of the displacement of the nodes in the elements to which it belongs, hence only particular shapes of elements are considered in the FE method, such as triangles or quandrangles in the case $d = 2$, and tetrahedra, hexahedra or prisms when $d = 3$.
      % \par
      % The HHO method seperates the two concepts of mesh and polynomial basis, by seeking an approximations of $\tensori{u}$ in a richer space than $\Hone$, namely in the (non-conformal) broken Sobolev space $\Hbroken \supset \Hone$. The physiscal interpretation is that the displacement field is enriched with a rigid body motion at the element level.
      % Since $\Hbroken$ is discontinuous, \eqref{eq_weak_problem} is considered at the element level, and the mesh of $\Omega$ provides just a discretization in the geometrical space, without any link with the discretization in functional space as in the FE method : any polynomial basis of $\Pbroken$ can be considered to ensure discretization function-wise at the element level, and the most convinient choice is to use the monomial basis (see \ref{sec_mesh}). The coordinates of any polynomial in $\Pbroken$ have no physical meaning (they are not the displacement at a given node as in the FE method).

      \begin{defbox}{Broken Sobolev space}
        \begin{equation}
          \Hbroken =
          \Bigg\{
            \tensori{u} \in L^2(\Omega;\Rd)
            \ \ \Bigg\vert \ \ 
            \forall T \in \Mesh(\Omega), \tensori{u} \vert_T \in H^1(T;\Rd)
          \Bigg\}
        \end{equation}
      \end{defbox}

      In other words, the broken Sobolev space is that of all piece-wise Sobolev space in every element $K$ in $\Mesh(\Omega)$ : hence, discontinuities or jumps across elements are allowed.

      \begin{infobox}{Mechanical interpretation of the broken Sobolev space}
        In mechanical terms, considerting the broken Sobolev space for the displacement $\tensori{u}$ 
      \end{infobox}

    \subsection{Admissible meshes}
    \label{sec_mesh}

      As mentioned above, contrary to the FE method, the HHO method separates the notions of mesh and that of polynomial basis (or shape function in the FE method) : the latter are not intrinsically linked to the mesh as it is the case with the FE method (see section \ref{sec_polynomial_basis}), which allows to consider a larger range of admssible meshes than those spanned by the FE method in the framework of the HHO method :

      \begin{defbox}{Admissible meshes}

        A mesh $\Mesh(\Omega)$ over $\Omega$ is said to be admissible if it is is a finite collection of \textbf{nonempty disjoint open convex polytopes} $T$ with \textbf{planar faces}, such that $\Omega = \cup_{T \in \Mesh(\Omega)} T$, and :

        \begin{equation}
          h = \sup_{T \in \Mesh(\Omega)} \{ h_T \} \mbox{ where $h_T$ denotes the diameter of $T$}
        \end{equation}

      \end{defbox}

      Since any polytopal domain is considered, elements in the HHO method can directly take the shape of any grain in a crystal for instance, giving a direct geometrical and physical interpretation to the mesh.
      
      \begin{exemplebox}{Admissible meshes}
        EXEMPLE DE MESH
      \end{exemplebox}
  
    \subsection{Polynomial basis}
    \label{sec_polynomial_basis}

        As mentioned above, the polynomial basis of $\Pbroken$ considered in the HHO method is not the Lagrange polynomial basis, but the monomial basis, which is the assembly of monomials in $\Pbroken$ (\textit{i.e.} the assembly of all $x \mapsto x^{\alpha}$ where $\alpha$ defines the power of the monomial). In addition, the monomial is sclaed with respect to the element in which it acts, so that values taken by the unknown at a given point in $\Omega$ are not influenced by the mesh topolgy.
        \par
        In order to properly define such a basis, we define the following exponents sets :

        \begin{defbox}{Scaled monomial exponents}

          Let $k \geq 0$ and $d \geq 1$ two integers. One defines the exponents vector set $\captize{\alpha}(d, k)$ : 

          \begin{equation}
            \begin{array}{lll}
              \captize{\alpha}(d, k) =
              \Bigg\{
                \captize{\alpha}_j(d)
                \ \ \Bigg\vert \ \ 
                0 \leq j \leq k
              \Bigg\}
              &
              \mbox{and}
              &
              \captize{\alpha}_k(d) =
              \Bigg\{
                \tensori{\alpha} = (\tensoro{\alpha}\subscript{1}, ..., \tensoro{\alpha}\subscript{d}) \in \mathbb{N}^d
                \ \ \Bigg\vert \ \ 
                \sum_{1 \leq i \leq d} \tensoro{\alpha}\subscript{i} = k
              \Bigg\}
            \end{array}
          \end{equation}

          % \begin{equation}
          %   \captize{\alpha}(d, k) =
          %   \Bigg\{
          %     \captize{\alpha}_j(d)
          %     \ \ \Bigg\vert \ \ 
          %     0 \leq j \leq k
          %   \Bigg\}
          % \end{equation}

        \end{defbox}
        
        \begin{exemplebox}{Scaled monomial exponents}

          For $k = 3$ and $d = 1$, $\captize{\alpha}_3(1) = \{ (3) \}$ and $\captize{\alpha}(1,3) = \{ (0), (1), (2),  (3) \}$

          For $k = 2$ and $d = 2$, $\captize{\alpha}_2(2) = \{ (0,2), (1,1), (2,0) \}$ and $\captize{\alpha}(2,2) = \{ (0,0), (0,1), (1,0), (0,2), (1,1), (2,0) \}$

        \end{exemplebox}

        $\captize{\alpha}_k(d)$ and $\captize{\alpha}(d,k)$ define exponents vectors sets that entirely define the size of the polynomial basis to be used. As mentioned above, the chosen polynomial basis is scaled, hence we define the scaled monomial basis with respect to a given domain in which it acts :

        \begin{defbox}{Scaled monomial basis}

          Let $1 \leq d \leq 3$ and $k \geq 0$ two integers. Let $\Domain \subset \Rd$ a closed domain of volume $\tensoro{v}_{\Domain}$ and of barycenter $\tensori{x} \subscript{\Domain}$. The (scaled) monomial basis $\mathcal{B}_{sm}(k, d)$ of polynbomials of order $k$ in $\Domain$ writes as :

          \begin{equation}
            \mathcal{B}_{sm}(k, d) =
            \Bigg\{
              \tensori{x} \mapsto
              \prod_{1 \leq i \leq d}
              {\left(
                \frac{\tensoro{x}\subscript{i} - \tensoro{x}\subscript{i} \subscript{\Domain}}{\tensoro{v}_{\Domain}}
              \right)}^{\tensoro{\alpha}\subscript{i}}
              \ \ \Bigg\vert \ \ 
              \tensori{\alpha} = (\tensoro{\alpha}\subscript{1}, ..., \tensoro{\alpha}\subscript{d}) \in \captize{\alpha}(d,k)
            \Bigg\}
          \end{equation}

          In particular, the dimension $N_k^d$ of $\mathcal{B}_{sm}(k, d)$ is $N_d^k = \binom{d+k}{k}$

        \end{defbox}

    \subsection{HHO elements}

        

    \bibliographystyle{unsrt}

  \bibliography{/Users/davidsiedel/Projects/Documents/Bibliography/bibliography.bib}
  


  \end{document}