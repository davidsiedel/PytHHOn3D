% \documentclass{article}
\documentclass[fleqn]{article}

\usepackage{notations}
\usepackage{amsfonts}
\usepackage{amsmath}
\usepackage{relsize}

% ======================================================================================================================
% NOTATIONS
% ======================================================================================================================

\newcommand{\Hdiv}{
  H^1_{div}(\Omega; \mathbb{R}^{d \times d})
}

\newcommand{\Hone}{
  H^{1}(\Omega; \mathbb{R}^{d})
}

\newcommand{\Fv}{
  \tensori{f} \subscript{v}
}

\newcommand{\Tn}{
  \tensori{t} \subscript{n}
}

\newcommand{\PK}{
  \tensorii{\pi} \subscript{1}
}

\newcommand{\Domain}{
  D
}

% ======================================================================================================================
% NOTATIONS
% ======================================================================================================================

% \setlength{\mathindent}{0pt}
\setlength{\parindent}{0em}
\setlength{\parskip}{0em}

\title{\texttt{PytHHOn3D} documentation}
\author{David Siedel \and Olivier Fandeur \and  Thomas Helfer}

\date{01/08/2020}


% ======================================================================================================================
% DOCUMENT
% ======================================================================================================================

\begin{document}

  \maketitle
  
  \section{Introduction}
      The \texttt{PytHHOn3D} library is a PDE solver implementing the HHO method in a generic and object-oriented fashion using \texttt{python}, with specific applications to solid mechanics : the structure of the code follows that of a usual FEM solver, namely :
      \begin{itemize}
        \item \texttt{PytHHOn3D} reads and parses a mesh file as input
        \item it takes a set of input data to set the PDE problem to solve (boundary conditions, volumetric forces, HHO elements characteristics)
        \item \texttt{PytHHOn3D} builds the framework specific to HHO methods, and the HHO element for each element in the mesh
        \item it solves a linear system given a set of fourth order tensors (one for each element) as tangent matrices
        \item it returns the unknown values at vertices and quadrature points
      \end{itemize}

  \newpage

  \section{Model problem : small strain linear elasticity}

    \paragraph{Problem setting}

      Let $\Omega$ a domain in $\Rd$ with Lipschitz boundaries $\partial \Omega$. Let the following regular functional spaces :

      \begin{defbox}{Sobolev space}
        $\Hone = \big\{ \tensori{u} \in L^2(\Omega;\mathbb{R}^{d}), \nabla \tensori{u} \in L^2(\Omega;\mathbb{R}^{d \times d}) \big\}$
      \end{defbox}

      \begin{defbox}{Divergence Sobolev space}
        $\Hdiv = \big\{ \tensorii{\tau} \in L^2(\Omega;\mathbb{R}^{d \times d}), \nabla \cdot \tensorii{\tau} \in L^2(\Omega;\mathbb{R}^{d}) \big\}$
      \end{defbox}

      % Let the Poisson problem :

      % \begin{equation}
      %   \begin{cases}
      %     \Delta \tensori{u} = \tensori{f} & \mbox{in } \Omega
      %     \\
      %     \nabla \tensori{u} = \tensori{t} \cdot \tensori{n} & \mbox{on } \partial \Omega_N
      %     \\
      %     \tensori{u} = \tensori{u} \subscript{d} & \mbox{on } \partial \Omega_D
      %   \end{cases}
      %   \label{eq_poisson}
      % \end{equation}

    \paragraph{Problem}

      Let the small strain linear elastic problem for $\tensori{u}$ such that:

      \begin{equation}
        \begin{cases}
          \nabla \cdot \tensorii{\sigma} = -\tensori{f} & \mbox{in } \Omega
          \\
          \tensorii{\sigma} = \tensori{t} \cdot \tensori{n} & \mbox{on } \partial \Omega_N
          \\
          \tensori{u} = \tensori{u} \subscript{D} & \mbox{on } \partial \Omega_D
          \\
          \tensorii{\sigma} = 2 \mu \tensorii{\varepsilon} + \lambda \Tr (\tensorii{\varepsilon}) = \tensoriv{\mathbb{C}} : \tensorii{\varepsilon} & \mbox{in } \Omega
          \\
          \tensorii{\varepsilon} = \frac{1}{2}(\nabla \tensori{u} + \nabla^T \tensori{u}) = \nabla^s \tensori{u} & \mbox{in } \Omega
        \end{cases}
        \label{eq_linear_elastic}
      \end{equation}
      
      Where

      \begin{itemize}
        \item $\tensorii{\sigma} \in \Hdiv$ is the symmetric Cauchy stress tensor
        \item $\tensori{u} \in \Hone$ is the displacement field
        \item $\tensori{t} \in \Hone$ are the applied forces
        \item $\tensori{u} \subscript{D} \in \Hone$ is the imposed displacement
        \item $\lambda$ and $\mu$ are Lamé coefficients of the material
      \end{itemize}

    \paragraph{Weak form}

      The weak form of \eqref{eq_linear_elastic} reads as the Principle of Virtual Works :

      \begin{theorembox}{Principle of Virtual Works (PVW)}
        find $\tensori{u} \in \Hone$ such that : 
        \begin{equation}
          \begin{cases}
            \displaystyle
            \int_{\Omega} \tensorii{\sigma} : \nabla^s \tensori{v} = \int_{\Omega} \tensori{f} \cdot \tensori{v} + \int_{\partial \Omega_N} \tensori{t} \cdot \tensori{v} &  \forall \tensori{v} \in \Hone
            \\
            \tensori{u} = \tensori{u} \subscript{D} & \mbox{on } \partial \Omega_D
          \end{cases}
        \end{equation}
        \label{eq_pvw}
      \end{theorembox}

    \paragraph{Discretization : the Finite Element Method (FEM)}

      The FEM method consists in descretizing \eqref{eq_pvw} in both physical and functional spaces through Lagrange polynbomials bases : defining the usual shape functions, that are polynomials of a given order $k_c$, such that they are of value $1$ at a given node and $0$ elsewhere in $\Omega$, one can put \eqref{eq_pvw} in a matricial form, which enables to find an approximation of $\tensori{u}$ in \eqref{eq_pvw} numerically.
    
    \paragraph{Polynomial approximation}

      Let $\mathbb{P}^{k_c}(\Omega;\mathbb{R}^d) \subset \Hone$ a polynomial space of order $k_c$. Let $N_k^d$ the dimension of $\mathbb{P}^{k_c}(\Omega;\mathbb{R}^d)$, and $(\tensori{\phi}\subscript{m})_{m \leq N_{k_c}^d}$ a basis of $\mathbb{P}^{k_c}(\Omega;\mathbb{R}^d)$. $N_k^d$ is then the number of vectors $\tensori{\phi}\subscript{m}$ composing 
    
  \section{The HHO method}
    
      \paragraph{Polynomial basis}

        Let the scaled polynomial basis

        \begin{defbox}{Scaled monomial exponents}

          Let $k_j \geq 0$ and $d \geq 1$ two integers.

          \begin{equation}
            \captize{\alpha}(k_j,d) =
            \Bigg\{
              \tensori{\alpha} = (\alpha_1, ..., \alpha_{d})
              \ \ \Bigg\vert \ \ 
              \sum_{1 \leq i \leq d} \alpha_i = k_j
            \Bigg\}
          \end{equation}

          For a given integer $k$, we define :

          \begin{equation}
            \captize{\alpha}_{mono}(k,d) =
            \Bigg\{
              \captize{\alpha}(k_j,d)
              \ \ \Bigg\vert \ \ 
              0 \leq k_j \leq k
            \Bigg\}
          \end{equation}

        \end{defbox}
        
        \begin{infobox}{Scaled monomial exponents}

          Let $k_j \geq 0$ and $d \geq 1$ two integers.

          \begin{equation}
            \captize{\alpha}(k_j,d) =
            \Bigg\{
              \tensori{\alpha} = (\alpha_1, ..., \alpha_{d})
              \ \ \Bigg\vert \ \ 
              \sum_{1 \leq i \leq d} \alpha_i = k_j
            \Bigg\}
          \end{equation}

          For a given integer $k$, we define :

          \begin{equation}
            \captize{\alpha}_{mono}(k,d) =
            \Bigg\{
              \captize{\alpha}(k_j,d)
              \ \ \Bigg\vert \ \ 
              0 \leq k_j \leq k
            \Bigg\}
          \end{equation}

        \end{infobox}

        \begin{defbox}{Scaled monomial basis}

          Let $\Domain \subset \Rd$ a closed domain of volume $\tensoro{v}_{\Domain}$ and of barycenter $\tensori{x} \subscript{\Domain}$. The (sclaed) monomial basis of polynbomials $\mathcal{B}_{mono}(k, d)$ of order $k$ in $\Domain$ writes as :

          \begin{equation}
            \mathcal{B}_{mono}(k, d) =
            \Bigg\{
              \prod_{\tensoro{\alpha} \in \scriptize{\tensori{\alpha}}}
              {\left(
                \frac{\tensori{x} - \tensori{x} \subscript{\Domain}}{\tensoro{v}_{\Domain}}
              \right)}^{\tensoro{\alpha}}
              \ \ \Bigg\vert \ \ 
              \tensori{\alpha} \in \captize{\alpha}_{mono}(k,d)
            \Bigg\}
          \end{equation}

          In particular, the dimension $N_k^d$ of $\mathcal{B}_{mono}(k, d)$ is $N_d^k = \binom{d+k}{k}$

        \end{defbox}

    \bibliographystyle{unsrt}

  \bibliography{/Users/davidsiedel/Projects/Documents/Bibliography/bibliography.bib}
  
\end{document}